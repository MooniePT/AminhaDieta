\documentclass[12pt, a4paper]{report}
\usepackage[utf8]{inputenc}
\usepackage[portuguese]{babel}
\usepackage[T1]{fontenc}
\usepackage{graphicx}
\usepackage{geometry}
\usepackage{hyperref}
\usepackage{listings}
\usepackage{xcolor}
\usepackage{acronym}
\usepackage{float}
\usepackage{titlesec}
\usepackage{setspace}

% Configuração das margens
\geometry{top=3cm, bottom=2.5cm, left=3cm, right=2.5cm}

% Configuração de Código Java
\definecolor{javapurple}{rgb}{0.5,0,0.35}
\definecolor{javadocblue}{rgb}{0.25,0.35,0.75}
\definecolor{javakeyword}{rgb}{0.5,0,0.35}
\definecolor{javastring}{rgb}{0.16,0.00,1.00}

\lstset{
  language=Java,
  basicstyle=\ttfamily\small,
  keywordstyle=\color{javakeyword}\bfseries,
  stringstyle=\color{javastring},
  commentstyle=\color{gray},
  morecomment=[s][\color{javadocblue}]{/**}{*/},
  numbers=left,
  numberstyle=\tiny\color{black},
  stepnumber=1,
  numbersep=10pt,
  tabsize=4,
  showspaces=false,
  showstringspaces=false,
  breaklines=true,
  frame=single,
  captionpos=b
}

% Capa
\title{
    \textbf{\LARGE UNIVERSIDADE DA BEIRA INTERIOR}\\
    \large Engenharia Informática - 2º Ano / 1º Semestre\\
    \large Programação Orientada a Objetos\\[2cm]
    \textbf{\Huge A Minha Dieta}\\
    \Large Sistema de Monitorização Nutricional em Java\\[3cm]
}
\author{\textbf{Carlos}\\[1cm]}
\date{Dezembro de 2025}

\begin{document}

\maketitle

% Agradecimentos
\chapter*{Agradecimentos}
Gostaria de expressar o meu sincero agradecimento a todos os que contribuíram para a realização deste projeto.

Em primeiro lugar, ao corpo docente da Unidade Curricular de Programação Orientada a Objetos da Universidade da Beira Interior, por fornecerem os conhecimentos fundamentais de arquitetura de software, padrões de desenho e linguagem Java, sem os quais este trabalho não teria sido possível.

Aos meus colegas de curso, pelas discussões estimulantes e pela partilha de conhecimentos e dúvidas que enriqueceram o meu processo de aprendizagem.

Por fim, agradeço à minha família e amigos pelo apoio constante e pela paciência demonstrada durante as longas horas de desenvolvimento e estudo dedicadas a este projeto prático.

% Resumo
\chapter*{Resumo}
O presente relatório descreve o desenvolvimento da aplicação "A Minha Dieta", um sistema de software desenvolvido em Java com interface gráfica JavaFX, destinado à gestão e monitorização da saúde nutricional dos utilizadores.

A aplicação foi concebida com o objetivo de permitir aos utilizadores registar o seu perfil biométrico, acompanhar a ingestão diária de alimentos e água, e visualizar o progresso em relação a metas de saúde estabelecidas automaticamente através de algoritmos reconhecidos cientificamente, como a Equação de Mifflin-St Jeor.

Ao longo deste documento, é detalhado todo o ciclo de vida do desenvolvimento, desde a análise de requisitos e escolha de tecnologias, passando pela arquitetura do sistema baseada no padrão Model-View-Controller (MVC), até à implementação concreta das funcionalidades de registo, cálculo de macronutrientes e visualização de dados. O projeto demonstra a aplicação prática de conceitos avançados de Programação Orientada a Objetos (POO), persistência de dados e design de interfaces modernas.

\textbf{Palavras-chave:} Java, JavaFX, POO, Monitorização de Saúde, MVC, Nutrição.

% Índices
\tableofcontents
\listoffigures
\lstlistoflistings

% Lista de Acrónimos
\chapter*{Lista de Acrónimos}
\begin{acronym}
    \acro{POO}{Programação Orientada a Objetos}
    \acro{MVC}{Model-View-Controller}
    \acro{GUI}{Graphical User Interface}
    \acro{IMC}{Índice de Massa Corporal}
    \acro{TMB}{Taxa Metabólica Basal}
    \acro{FXML}{FX Markup Language}
    \acro{CSS}{Cascading Style Sheets}
    \acro{API}{Application Programming Interface}
    \acro{IDE}{Integrated Development Environment}
\end{acronym}

% Capítulo 1: Introdução
\chapter{Introdução}
\section{Contextualização}
A monitorização da saúde e da nutrição tornou-se uma preocupação central na sociedade moderna. Com o aumento de doenças relacionadas com o estilo de vida, como a obesidade e a diabetes, a necessidade de ferramentas que auxiliem os indivíduos a gerir a sua alimentação e hidratação é cada vez mais premente. A tecnologia desempenha, neste contexto, um papel facilitador fundamental, transformando cálculos metabólicos complexos em interfaces simples e intuitivas.

Este projeto, intitulado "A Minha Dieta", surge no âmbito da disciplina de \ac{POO} do curso de Engenharia Informática. O desafio proposto foi o de conceber e implementar uma aplicação desktop robusta, capaz de gerir dados de múltiplos utilizadores e persistir essa informação entre sessões.

\section{Motivação}
A escolha de desenvolver uma aplicação de nutrição prende-se com a sua aplicabilidade prática e a complexidade interessante que oferece em termos de modelação de dados. Requer não apenas estruturas de dados para armazenar perfis e históricos, mas também algoritmos para calcular necessidades biológicas dinâmicas.

\section{Estrutura do Relatório}
Este relatório estrutura-se da seguinte forma:
\begin{itemize}
    \item \textbf{Capítulo 2: Objetivos} - Define as metas do projeto.
    \item \textbf{Capítulo 3: Enquadramento Teórico} - Explora as tecnologias e conceitos usados.
    \item \textbf{Capítulo 4: Desenvolvimento} - Detalha a implementação técnica.
    \item \textbf{Capítulo 5: O Projeto} - Manual de utilização.
    \item \textbf{Capítulo 6: Testes} - Validação funcional.
    \item \textbf{Capítulo 7: Conclusão} - Reflexão final.
\end{itemize}

% Capítulo 2: Objetivos
\chapter{Objetivos}
O objetivo principal deste projeto é desenvolver uma aplicação de gestão dietética funcional, intuitiva e esteticamente agradável, aplicando corretamente os paradigmas da \ac{POO}.

\section{Objetivos Gerais}
\begin{itemize}
    \item Aplicar conceitos de Encapsulamento, Herança, Polimorfismo e Abstração.
    \item Implementar uma arquitetura de software organizada (\ac{MVC}).
    \item Garantir a persistência de dados através de serialização de objetos.
    \item Criar uma \ac{GUI} responsiva e amigável.
\end{itemize}

\section{Objetivos Específicos}
\begin{itemize}
    \item \textbf{Gestão de Perfis}: Permitir o registo, edição e autenticação de múltiplos utilizadores.
    \item \textbf{Cálculos Automáticos}: Implementar o cálculo automático do \ac{IMC}, \ac{TMB} e necessidades hídricas diárias.
    \item \textbf{Monitorização Alimentar}: Possibilitar o registo de refeições manuais ou via base de dados de alimentos pré-definidos.
    \item \textbf{Gestão de Macronutrientes}: Rastrear Proteínas, Hidratos de Carbono e Gorduras.
    \item \textbf{Histórico}: Apresentar visualmente a evolução do peso e histórico de refeições.
\end{itemize}

% Capítulo 3: Enquadramento Teórico
\chapter{Enquadramento Teórico}

\section{Engenharia de Software e o Padrão MVC}
Para garantir a manutenibilidade e escalabilidade do código, o projeto segue o padrão de arquitetura \ac{MVC}.
\begin{description}
    \item[Model (Modelo)] Representa os dados e a lógica de negócio (ex: \texttt{UserProfile}, \texttt{MealEntry}). Estas classes desconhecem a interface gráfica.
    \item[View (Vista)] Responsável pela apresentação visual. Em JavaFX, definida por ficheiros \ac{FXML} (ex: \texttt{HomeView.fxml}).
    \item[Controller (Controlador)] Intermediário que processa inputs da Vista e atualiza o Modelo (ex: \texttt{HomeController.java}).
\end{description}

\section{Java e JavaFX}
O projeto utiliza \textbf{Java 21} e \textbf{JavaFX}. O JavaFX introduz o conceito de \textit{Scene Graph}, separação de design (\ac{FXML}/\ac{CSS}) e lógica, suporte a propriedades observáveis e \textit{data binding}, cruciais para interfaces reativas.

\section{Conceitos Nutricionais}
A aplicação implementa fórmulas científicas:
\begin{itemize}
    \item \textbf{\ac{IMC}}: $Peso (kg) / Altura (m)^2$.
    \item \textbf{Equação de Mifflin-St Jeor (TMB)}:
    \begin{itemize}
        \item Homens: $(10 \times peso) + (6.25 \times altura) - (5 \times idade) + 5$
        \item Mulheres: $(10 \times peso) + (6.25 \times altura) - (5 \times idade) - 161$
    \end{itemize}
    \item \textbf{Macronutrientes}:
    \begin{itemize}
        \item Proteína: 4 kcal/g
        \item Hidratos: 4 kcal/g
        \item Gordura: 9 kcal/g
    \end{itemize}
\end{itemize}

% Capítulo 4: Desenvolvimento e Implementação
\chapter{Desenvolvimento e Implementação}

\section{Análise de Requisitos}
Foram definidos requisitos funcionais (RF) e não-funcionais (RNF):
\begin{itemize}
    \item \textbf{RF01}: Criar, editar e selecionar perfis.
    \item \textbf{RF02}: Calcular metas calóricas e hídricas automaticamente.
    \item \textbf{RF03}: Registar alimentos com info nutricional.
    \item \textbf{RF04}: Registar refeições (data/hora).
    \item \textbf{RNF01}: Resolução mínima 1280x800.
    \item \textbf{RNF02}: Tema visual "Fresh \& Healthy".
\end{itemize}

\section{Arquitetura do Sistema}
A estrutura de pacotes separa responsabilidades:
\begin{itemize}
    \item \texttt{app.model}: Classes de domínio.
    \item \texttt{app.ui}: Gestão de cenas (\texttt{SceneManager}).
    \item \texttt{app.ui.controller}: Lógica de interação.
    \item \texttt{app.persistence}: Gestão de ficheiros (\texttt{DataStore}).
\end{itemize}

\section{Modelação de Dados}
A classe \texttt{UserProfile} é central, agregando listas de registos.

\begin{lstlisting}[caption={Excertos da classe UserProfile}, label={lst:userprofile}]
public class UserProfile implements Serializable {
    private String nome;
    private int idade;
    private double pesoKg;
    // ...
    private List<MealEntry> meals = new ArrayList<>();
    private List<Food> foods = new ArrayList<>(); 

    public double getBMI() { // Cálculo de IMC
        double alturaM = alturaCm / 100.0;
        return pesoKg / (alturaM * alturaM);
    }
    
    public int getCaloriesConsumedToday() {
        // Stream API para somar calorias do dia
        return meals.stream()
            .filter(m -> m.isToday())
            .mapToInt(MealEntry::getCalories)
            .sum();
    }
}
\end{lstlisting}

A classe \texttt{MealEntry} foi expandida para suportar Macronutrientes.

\begin{lstlisting}[caption={Classe MealEntry}, label={lst:mealentry}]
public class MealEntry implements Serializable {
    private String description;
    private int calories;
    private double protein; // Gramas
    private double carbs;
    private double fat;
    // ... construtores e getters
}
\end{lstlisting}

\section{Interface de Utilizador}
O \texttt{SceneManager} controla a navegação, permitindo uma experiência de "Single Page Application" ao trocar apenas o conteúdo central do \texttt{DashboardView}.

\begin{lstlisting}[caption={Método de navegação no SceneManager}, label={lst:scenemanager}]
public void showDashboard() {
    if (appState.getActiveProfile() == null) {
        showLogin();
        return;
    }
    // Carrega o FXML do Dashboard e injeta o perfil ativo
    loadScene("/fxml/DashboardView.fxml", "A Minha Dieta - Dashboard");
}
\end{lstlisting}

\section{Gestão de Macronutrientes}
No \texttt{HomeController}, as barras de progresso são atualizadas dinamicamente.

\begin{lstlisting}[caption={Atualização das Barras de Progresso}, label={lst:homecontroller}]
private void updateMacro(Label label, ProgressBar bar, double current, double goal, String unit) {
    label.setText(String.format("%.0f%s / %.0f%s", current, unit, goal, unit));
    bar.setProgress(goal > 0 ? current / goal : 0);
}
\end{lstlisting}

% Capítulo 5: O Projeto
\chapter{Manual de Utilização}

\section{Início e Autenticação}
Ao iniciar, o utilizador encontra o ecrã de \textbf{Login}.
\begin{enumerate}
    \item Se não houver perfis, é redirecionado para o \textbf{Registo}.
    \item No registo, insere dados biométricos e seleciona o género.
    \item Clica em "Criar Perfil" para aceder ao Dashboard.
\end{enumerate}

\section{Dashboard}
O painel principal oferece uma visão geral:
\begin{itemize}
    \item \textbf{Resumo Calórico}: Barra verde indicando consumo vs meta.
    \item \textbf{Macronutrientes}: Barras para Proteína, Hidratos e Gordura.
    \item \textbf{Gráfico de Peso}: Evolução temporal do peso do utilizador.
\end{itemize}

\section{Gestão de Refeições}
Organizada em duas abas:
\begin{enumerate}
    \item \textbf{Diário}: Registo de refeições. Permite preenchimento manual ou seleção rápida via combobox "Pré-definidos".
    \item \textbf{Meus Alimentos}: Criação de alimentos personalizados (ex: "Aveia", "Frango") que ficam guardados para uso futuro.
\end{enumerate}

% Capítulo 6: Testes
\chapter{Testes e Validação}
Foram realizados testes manuais para validar as funcionalidades.

\section{Cenários de Teste}

\begin{table}[H]
\centering
\begin{tabular}{|p{2cm}|p{4cm}|p{4cm}|p{3cm}|}
\hline
\textbf{ID} & \textbf{Cenário} & \textbf{Resultado Esperado} & \textbf{Obs.} \\
\hline
T01 & Input Numérico Inválido & Mensagem de erro apropriada & Sucesso \\
\hline
T02 & Persistência de Dados & Dados mantidos após reinício & Sucesso \\
\hline
T03 & Cálculo de Metas & Metas alteram com mudança de peso & Sucesso \\
\hline
T04 & Seleção de Alimento & Campos auto-preenchidos na ref. & Sucesso \\
\hline
\end{tabular}
\caption{Tabela de Casos de Teste}
\end{table}

\section{Análise de Resultados}
A aplicação comportou-se de forma estável. A introdução da validação de tipos (ex: impedir letras em campos de peso) preveniu erros de execução comuns. A persistência garantida pela serialização Java funcionou corretamente em Windows.

% Capítulo 7: Conclusão
\chapter{Conclusão}
\section{Balanço Final}
O projeto "A Minha Dieta" permitiu consolidar conhecimentos de \ac{POO} e \ac{GUI}. A construção de uma aplicação completa, com múltiplas vistas e gestão de estado complexa, ofereceu desafios reais de engenharia de software que foram superados através de um planeamento cuidado e arquitetura modular.

\section{Trabalhos Futuros}
\begin{itemize}
    \item Migração para base de dados SQL.
    \item Sincronização Cloud/API REST.
    \item Gamificação (medalhas por objetivos cumpridos).
\end{itemize}

% Bibliografia
\begin{thebibliography}{9}
\bibitem{java}
Oracle. \textit{Java Platform Standard Edition Documentation}. 2025.
\bibitem{openjfx}
OpenJFX. \textit{JavaFX Documentation}. https://openjfx.io/
\bibitem{mifflin}
Mifflin, M. D., et al. \textit{A new predictive equation for resting energy expenditure}. 1990.
\end{thebibliography}

\end{document}
