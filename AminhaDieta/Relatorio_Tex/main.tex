\documentclass[12pt, a4paper]{report}
\usepackage[utf8]{inputenc}
\usepackage[portuguese]{babel}
\usepackage[T1]{fontenc}
\usepackage{graphicx}
\usepackage{geometry}
\usepackage{hyperref}
\usepackage{listings}
\usepackage{xcolor}
\usepackage{acronym}
\usepackage{float}
\usepackage{titlesec}
\usepackage{setspace}
\usepackage{booktabs}
\usepackage{tabularx}

% Configuração das margens
\geometry{top=3cm, bottom=2.5cm, left=3cm, right=2.5cm}

% Configuração de Código Java
\definecolor{javapurple}{rgb}{0.5,0,0.35}
\definecolor{javadocblue}{rgb}{0.25,0.35,0.75}
\definecolor{javakeyword}{rgb}{0.5,0,0.35}
\definecolor{javastring}{rgb}{0.16,0.00,1.00}

\lstset{
  language=Java,
  basicstyle=\ttfamily\small,
  keywordstyle=\color{javakeyword}\bfseries,
  stringstyle=\color{javastring},
  commentstyle=\color{gray},
  morecomment=[s][\color{javadocblue}]{/**}{*/},
  numbers=left,
  numberstyle=\tiny\color{black},
  stepnumber=1,
  numbersep=10pt,
  tabsize=4,
  showspaces=false,
  showstringspaces=false,
  breaklines=true,
  frame=single,
  captionpos=b,
  literate={á}{{\'a}}1 {ã}{{\~a}}1 {é}{{\'e}}1 {ç}{{\c{c}}}1 {í}{{\'i}}1 {ó}{{\'o}}1 {õ}{{\~o}}1 {ú}{{\'u}}1
}

% Capa
\title{
    \textbf{\LARGE UNIVERSIDADE DA BEIRA INTERIOR}\\
    \large Engenharia Informática - 2º Ano / 1º Semestre\\
    \large Programação Orientada a Objetos\\[2cm]
    \textbf{\Huge A Minha Dieta}\\
    \Large Sistema de Monitorização Nutricional em Java\\[3cm]

    \vspace{0.07cm}
    \begin{center}
    \begin{normalsize}
    \begin{large}
    Elaborado por:
    \end{large}
    \end{normalsize}
    \end{center}
    
    \vspace{0.1cm}
    \begin{center}
    \begin{large}
    \textbf{Carlos Farinha, João Rodrigues, André Schroder, Henrique Marques}
    \end{large}
    \end{center}
}
\date{Dezembro de 2025}

\begin{document}

\maketitle

% Agradecimentos
\chapter*{Agradecimentos}
Gostaria de expressar o meu sincero agradecimento ao corpo docente da Unidade Curricular de Programação Orientada a Objetos da Universidade da Beira Interior, pelo rigor técnico exigido e pelos conhecimentos transmitidos sobre arquitetura de software e padrões de desenho. Agradeço também aos meus colegas pelas discussões técnicas que enriqueceram a implementação deste projeto.

% Resumo Executivo
\chapter*{Resumo Executivo}
Este relatório detalha o ciclo de vida de desenvolvimento da aplicação "A Minha Dieta", uma solução desktop robusta desenvolvida em Java e JavaFX para monitorização nutricional.

O projeto distingue-se pela implementação rigorosa do padrão \textbf{Model-View-Controller (MVC)}, garantindo uma separação clara entre a lógica de negócio e a interface gráfica. Funcionalidades avançadas incluem o cálculo metabólico dinâmico baseado na equação de Mifflin-St Jeor, gestão customizável de base de dados de alimentos e persistência de dados via serialização.

A aplicação supera os requisitos funcionais propostos, integrando uma experiência de utilizador fluida com validação de dados em tempo real e visualização gráfica de progresso, cumprindo todos os objetivos pedagógicos da unidade curricular.

\textbf{Palavras-chave:} Java, JavaFX, Arquitetura de Software, MVC, Engenharia de Requisitos.

% Índices
\tableofcontents
\listoffigures
\lstlistoflistings
\listoftables

% Acrónimos
\chapter*{Lista de Acrónimos}
\begin{acronym}
    \acro{POO}{Programação Orientada a Objetos}
    \acro{MVC}{Model-View-Controller}
    \acro{GUI}{Graphical User Interface}
    \acro{IMC}{Índice de Massa Corporal}
    \acro{TMB}{Taxa Metabólica Basal}
    \acro{BMR}{Basal Metabolic Rate}
    \acro{DTO}{Data Transfer Object}
    \acro{FXML}{FX Markup Language}
    \acro{JVM}{Java Virtual Machine}
\end{acronym}

% Capítulo 1: Introdução
\chapter{Introdução}
\section{Contextualização do Problema}
A obesidade e doenças metabólicas representam desafios significativos de saúde pública. A tecnologia pessoal desempenha um papel crucial na mitigação destes problemas, permitindo a auto-monitorização. O projeto "A Minha Dieta" visa preencher esta lacuna com uma ferramenta acessível, focada na precisão dos cálculos nutricionais.

\section{Objetivos}
O principal objetivo é o desenvolvimento de um sistema de software que aplique os pilares da \ac{POO}:
\begin{itemize}
    \item \textbf{Encapsulamento}: Proteção do estado interno dos objetos (\texttt{private fields}).
    \item \textbf{Abstração}: Simplificação de conceitos complexos (ex: \texttt{MealEntry}).
    \item \textbf{Herança e Polimorfismo}: Utilizados na estruturação de componentes gráficos e modelos de dados.
\end{itemize}

\section{Escopo}
A aplicação foca-se em utilizadores individuais (Single-User per session), mas suporta múltiplos perfis persistentes no mesmo dispositivo.

% Capítulo 2: Decisões Arquiteturais
\chapter{Decisões Arquiteturais e Tecnológicas}

\section{JavaFX vs Swing}
A escolha recaiu sobre o \textbf{JavaFX} devido à sua arquitetura moderna baseada em \textit{Scene Graph}. Ao contrário do Swing (single-threaded UI painting complexo), o JavaFX oferece:
\begin{enumerate}
    \item \textbf{Separação de Preocupações}: O uso de FXML separa a interface declarativa da lógica imperativa.
    \item \textbf{Styling CSS}: Capacidade de aplicar temas visuais ("Fresh \& Healthy") sem acoplamento ao código Java.
    \item \textbf{Data Binding}: Sincronização automática entre UI e Modelo (Properties).
\end{enumerate}

\section{Padrões de Desenho (Design Patterns)}
A robustez do código é garantida pela aplicação de padrões comprovados.

\subsection{Model-View-Controller (MVC)}
\begin{description}
    \item[Model] Classes de domínio puras (\texttt{UserProfile}, \texttt{MealEntry}). Ignoram a existência da UI, facilitando testes unitários.
    \item[View] Definida em ficheiros \texttt{.fxml}.
    \item[Controller] Classes que mediam a interação (ex: \texttt{MealsController}).
\end{description}

\subsection{Singleton (Adaptado) e Injeção de Dependência}
Para evitar variáveis globais estáticas (anti-pattern), a aplicação instancia o estado (\texttt{AppState}) uma única vez no \texttt{Main}. Este estado é então injetado ("passado para baixo") pelo \texttt{SceneManager} para cada controlador.
\begin{lstlisting}[language=Java, caption={Injeção de Dependência no SceneManager}]
// O SceneManager injeta 'state' e 'store' em cada controlador
controller.init(this, state, store);
\end{lstlisting}

\subsection{Data Transfer Object (DTO)}
As classes \texttt{Food} e \texttt{MealEntry} atuam como DTOs serializáveis, transportando dados entre a memória e o subsistema de armazenamento.

% Capítulo 3: Implementação Técnica
\chapter{Implementação Técnica Detalhada}

\section{O Problema do Launcher (Bootstrapping)}
Em ambientes modernos (Java 9+), executar uma classe que estende \texttt{Application} diretamente pode causar erros de carregamento de bibliotecas nativas se não estiver num módulo.
\textbf{Solução}: Implementou-se a classe intermédia \texttt{app.Launcher}, que não estende \texttt{Application}, forçando a \ac{JVM} a carregar o \textit{classpath} completo antes de inicializar o runtime JavaFX.

\section{Gestão de Cenas (SceneManager)}
Centralizou-se a navegação numa classe dedicada, evitando duplicação de loaders FXML.
\begin{itemize}
    \item \texttt{showLogin()}: Primeiro ponto de contacto.
    \item \texttt{showDashboard()}: Carrega o utilizador ativo.
\end{itemize}

\section{Persistência de Dados}
A persistência utiliza \texttt{java.io.Serializable}. O objeto raiz \texttt{AppState} contém uma lista de utilizadores.
\begin{lstlisting}[language=Java, caption={Estratégia Atomic Save}]
// Sempre que há uma alteração critica
user.addMeal(meal);
store.save(state); // Gravação imediata para prevenir perda de dados
\end{lstlisting}

% Capítulo 4: Algoritmos e Lógica
\chapter{Algoritmos e Lógica de Negócio}

\section{Cálculo Metabólico (Mifflin-St Jeor)}
A classe \texttt{UserProfile} encapsula a lógica matemática. A fórmula de Mifflin-St Jeor é considerada a mais precisa para estimativa de TMB.

\begin{equation}
TMB = (10 \times peso) + (6.25 \times altura) - (5 \times idade) + S
\end{equation}
Onde $S$ é $+5$ para homens e $-161$ para mulheres.

\section{Metas de Macronutrientes}
As metas são calculadas dinamicamente com base num rácio equilibrado:
\begin{itemize}
    \item Proteína (20\%): $Kcal \times 0.20 / 4 g$.
    \item Hidratos (50\%): $Kcal \times 0.50 / 4 g$.
    \item Gordura (30\%): $Kcal \times 0.30 / 9 g$.
\end{itemize}

\section{Filtragem de Listas}
Para exibir as refeições de "Hoje", o sistema filtra a lista completa em tempo de execução ($\mathcal{O}(N)$), o que é aceitável para o volume de dados esperado num contexto pessoal.

% Capítulo 5: Matriz de Rastreabilidade
\chapter{Matriz de Rastreabilidade}
A tabela seguinte mapeia os Requisitos Funcionais (RF) aos componentes de código implementados.

\begin{table}[H]
\centering
\begin{tabularx}{\textwidth}{|l|X|X|}
\hline
\textbf{ID} & \textbf{Requisito} & \textbf{Componente(s) de Implementação} \\
\hline
RF01 & Múltiplos Perfis & \texttt{LoginController}, \texttt{AppState} \\
\hline
RF02 & Cálculo Automático TMB & \texttt{UserProfile.getDailyCalorieGoal()} \\
\hline
RF03 & Registo de Refeições & \texttt{MealEntry}, \texttt{MealsController} \\
\hline
RF04 & Base de Dados Alimentos & \texttt{Food}, \texttt{FoodDatabaseController} \\
\hline
RF05 & Visualização Gráfica & \texttt{HomeView.fxml} (Charts), \texttt{HomeController} \\
\hline
RNF01 & Persistência & \texttt{DataStore} (File I/O) \\
\hline
\end{tabularx}
\caption{Matriz de Rastreabilidade de Requisitos}
\end{table}

% Capítulo 6: Testes e Validação
\chapter{Testes e Validação}

\section{Metodologia}
Foram realizados testes de caixa negra (funcionais) para validar os fluxos de utilizador.

\section{Cenários Críticos}
\begin{enumerate}
    \item \textbf{Input Validation}: Inserir texto em campos numéricos gera exceções tratadas (\texttt{try-catch}), exibindo mensagens amigáveis na UI.
    \item \textbf{Fluxo de "Meus Alimentos"}: Criar um alimento e usá-lo imediatamente no diário valida a integração entre controladores embebidos (\texttt{fx:include}).
    \item \textbf{Consistência de Dados}: Reiniciar a aplicação manteve o histórico intacto, validando a serialização.
\end{enumerate}

% Capítulo 7: Conclusão
\chapter{Conclusão e Trabalhos Futuros}
O projeto "A Minha Dieta" cumpre integralmente os requisitos da unidade curricular, entregando um produto de software robusto e bem arquitetado. A utilização de JavaFX permitiu elevar o nível da interface gráfica, proporcionando uma experiência superior à de aplicações Swing tradicionais.

\section{Trabalhos Futuros}
Para uma versão 2.0, sugere-se:
\begin{itemize}
    \item \textbf{Base de Dados SQL}: Migrar de ficheiros para SQLite para permitir queries complexas de histórico.
    \item \textbf{Exportação PDF}: Gerar relatórios mensais para impressão.
    \item \textbf{Sync Cloud}: Sincronização via API REST.
\end{itemize}

\end{document}
