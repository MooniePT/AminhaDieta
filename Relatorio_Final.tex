\documentclass[12pt,a4paper]{article}
\usepackage[utf8]{inputenc}
\usepackage[portuguese]{babel}
\usepackage{graphicx}
\usepackage{geometry}
\usepackage{hyperref}
\usepackage{listings}
\usepackage{xcolor}
\usepackage{float}
\usepackage{titlesec}
\usepackage{array}
\usepackage{booktabs}

% Configuração de Margens
\geometry{
 a4paper,
 total={170mm,257mm},
 left=25mm,
 top=25mm,
 right=25mm,
 bottom=25mm,
}

% Configuração de Código
\definecolor{codegreen}{rgb}{0,0.6,0}
\definecolor{codegray}{rgb}{0.5,0.5,0.5}
\definecolor{codepurple}{rgb}{0.58,0,0.82}
\definecolor{backcolour}{rgb}{0.95,0.95,0.92}

\lstdefinestyle{mystyle}{
    backgroundcolor=\color{backcolour},   
    commentstyle=\color{codegreen},
    keywordstyle=\color{magenta},
    numberstyle=\tiny\color{codegray},
    stringstyle=\color{codepurple},
    basicstyle=\ttfamily\footnotesize,
    breakatwhitespace=false,         
    breaklines=true,                 
    captionpos=b,                    
    keepspaces=true,                 
    numbers=left,                    
    numbersep=5pt,                  
    showspaces=false,                
    showstringspaces=false,
    showtabs=false,                  
    tabsize=2
}
\lstset{style=mystyle}

\title{
    \vspace{3cm}
    \Huge \textbf{A Minha Dieta} \\
    \vspace{1cm}
    \Large Relatório de Projeto Final \\
    \large Programação Orientada a Objetos \\
    \vspace{2cm}
    \includegraphics[width=0.4\textwidth]{AminhaDieta/src/main/resources/images/icon.png} % Placeholder se existir
    \vspace{3cm}
}

\author{
    \textbf{Equipa de Desenvolvimento:} \\
    Carlos Farinha \\
    João Rodrigues \\
    Henrique Marques \\
    André Schroder
}

\date{29 de Dezembro de 2024}

\begin{document}

\maketitle
\thispagestyle{empty}
\newpage

\section*{Agradecimentos}
Gostaríamos de expressar a nossa gratidão ao corpo docente da unidade curricular de Programação Orientada a Objetos por nos ter proporcionado os conhecimentos e o desafio necessários para a realização deste projeto. 
Agradecemos também a todos os colegas que, direta ou indiretamente, contribuíram com sugestões e feedback durante o desenvolvimento da aplicação ``A Minha Dieta''.
Por fim, um agradecimento especial a cada membro desta equipa pelo empenho, dedicação e espírito de colaboração demonstrados ao longo de todo o semestre.

\newpage

\tableofcontents
\newpage

\section{Introdução}
\subsection{Contextualização}
Na sociedade atual, a preocupação com a saúde e o bem-estar tem vindo a crescer exponencialmente. A gestão de uma dieta equilibrada, a prática regular de exercício físico e uma hidratação adequada são pilares fundamentais para um estilo de vida saudável. No entanto, o ritmo de vida acelerado torna difícil para muitas pessoas monitorizar estes aspetos de forma consistente.

\subsection{Objetivos do Projeto}
O projeto ``A Minha Dieta'' surge com o objetivo de colmatar esta necessidade, fornecendo uma aplicação desktop intuitiva e robusta que permite aos utilizadores:
\begin{itemize}
    \item Calcular as suas necessidades calóricas diárias com base em dados antropométricos.
    \item Registar e monitorizar a ingestão de alimentos e macronutrientes.
    \item Acompanhar o consumo diário de água com metas personalizadas.
    \item Registar atividades físicas e contabilizar as calorias queimadas.
    \item Visualizar o progresso através de relatórios e gráficos detalhados.
    \item Exportar relatórios em formato PDF.
    \item Personalizar a aparência visual da aplicação.
\end{itemize}

\subsection{Estrutura do Relatório}
Este relatório está organizado da seguinte forma:
\begin{itemize}
    \item \textbf{Cronograma}: Detalha as fases de desenvolvimento do projeto.
    \item \textbf{Arquitetura}: Explica as decisões de design e o padrão MVC.
    \item \textbf{Implementação}: Analisa em detalhe as classes e métodos desenvolvidos.
    \item \textbf{Funcionalidades}: Descreve todas as funcionalidades implementadas.
    \item \textbf{Testes}: Descreve a estratégia de validação do software.
    \item \textbf{Conclusão e Trabalho Futuro}: Reflete sobre o resultado final.
\end{itemize}

\newpage

\section{Cronograma de Desenvolvimento}
O desenvolvimento do projeto decorreu ao longo de várias semanas, seguindo uma metodologia ágil adaptada ao contexto académico.

\subsection{Fase 1: Planeamento e Análise (Semana 1-2)}
\begin{itemize}
    \item Definição dos requisitos funcionais e não funcionais.
    \item Esboço da interface gráfica (Mockups).
    \item Modelação do domínio (Diagrama de Classes inicial).
\end{itemize}

\subsection{Fase 2: Implementação do Core (Semana 3-4)}
\begin{itemize}
    \item Criação da estrutura do projeto Maven.
    \item Implementação das classes de modelo (\texttt{UserProfile}, \texttt{Food}, \texttt{MealEntry}).
    \item Implementação do sistema de persistência de dados (\texttt{DataStore}).
\end{itemize}

\subsection{Fase 3: Interface Gráfica e Lógica (Semana 5-6)}
\begin{itemize}
    \item Desenvolvimento das vistas FXML (Login, Dashboard, Refeições).
    \item Implementação dos Controladores JavaFX.
    \item Integração da lógica de negócio com a interface.
\end{itemize}

\subsection{Fase 4: Refinamento e Testes (Semana 7-8)}
\begin{itemize}
    \item Melhorias estéticas (CSS, ícones, gradientes).
    \item Implementação de funcionalidades avançadas (Gráficos, Exportação PDF).
    \item Testes manuais e correção de bugs.
    \item Documentação e elaboração do relatório final.
\end{itemize}

\subsection{Fase 5: Polimento Visual e Personalização (Semana 9)}
\begin{itemize}
    \item \textbf{Redesign da Interface}: Adoção de uma paleta de cores vibrante (gradiente Azul/Ciano) e tipografia moderna (Verdana).
    \item \textbf{Menu de Definições}: Implementação de um ecrã de configurações com:
    \begin{itemize}
        \item Toggle para Modo Arco-íris (animação dinâmica de gradiente HSB)
        \item ColorPicker para cor de fundo estática personalizada
        \item ComboBox para seleção de tipo de letra (Verdana, Arial, Segoe UI, Tahoma, Comic Sans MS)
    \end{itemize}
    \item \textbf{Gamificação}: Introdução de mensagens motivacionais (Alertas JavaFX) ao atingir metas de hidratação e registar exercícios.
    \item \textbf{Correções Visuais}: Ajuste de layouts e remoção de artefactos visuais nos componentes JavaFX.
\end{itemize}

\subsection{Fase 6: Exercício Físico e Análise (Semana 10)}
\begin{itemize}
    \item \textbf{Módulo de Exercício Físico}: Criação da classe \texttt{ExerciseEntry} e \texttt{ExerciseController}.
    \item \textbf{Gráfico de Exercício Semanal}: BarChart com calorias queimadas nos últimos 7 dias.
    \item \textbf{Tipos Pré-definidos}: Caminhada, Corrida, Ciclismo, Natação, Musculação, Yoga.
    \item \textbf{Atalhos de Alimentos}: Botões rápidos para Arroz, Massa, Batata, Leite, Ovos, Pão.
    \item \textbf{Base de Dados de Alimentos}: Módulo \texttt{FoodDatabaseController} para gestão de alimentos.
\end{itemize}

\newpage

\section{Arquitetura do Sistema}

\subsection{Padrão MVC (Model-View-Controller)}
A aplicação foi desenhada seguindo rigorosamente o padrão MVC, o que garante uma separação clara de responsabilidades e facilita a manutenção do código.

\subsubsection{Model (Modelo)}
O Modelo representa os dados e a lógica de negócio da aplicação. É independente da interface gráfica.

\begin{table}[H]
\centering
\begin{tabular}{@{}ll@{}}
\toprule
\textbf{Classe} & \textbf{Descrição} \\
\midrule
\texttt{UserProfile} & Perfil do utilizador com dados pessoais e registos \\
\texttt{AppState} & Estado global com lista de perfis e perfil ativo \\
\texttt{Food} & Alimento com informação nutricional por 100g \\
\texttt{MealEntry} & Registo de refeição com timestamp e macros \\
\texttt{WaterEntry} & Registo de consumo de água \\
\texttt{WeightEntry} & Registo histórico de peso \\
\texttt{ExerciseEntry} & Registo de exercício físico \\
\bottomrule
\end{tabular}
\caption{Classes do Modelo}
\end{table}

\subsubsection{View (Vista)}
A Vista é responsável pela apresentação dos dados ao utilizador. Utilizamos a tecnologia \textbf{JavaFX}, definindo a estrutura visual em ficheiros \texttt{.fxml} e o estilo em ficheiros \texttt{.css}.

\textbf{Vistas FXML implementadas:}
\begin{itemize}
    \item \texttt{LoginView.fxml} - Seleção de perfil
    \item \texttt{RegisterView.fxml} - Criação/edição de perfil
    \item \texttt{DashboardView.fxml} - Navegação principal
    \item \texttt{HomeView.fxml} - Dashboard com resumos
    \item \texttt{MealsView.fxml} - Registo de refeições
    \item \texttt{HydrationView.fxml} - Monitorização de água
    \item \texttt{ExerciseView.fxml} - Registo de exercícios
    \item \texttt{HistoryView.fxml} - Histórico e exportação
    \item \texttt{SettingsView.fxml} - Definições visuais
    \item \texttt{FoodDatabaseView.fxml} - Gestão de alimentos
\end{itemize}

\subsubsection{Controller (Controlador)}
O Controlador atua como intermediário entre a Vista e o Modelo.

\begin{table}[H]
\centering
\begin{tabular}{@{}ll@{}}
\toprule
\textbf{Controlador} & \textbf{Responsabilidade} \\
\midrule
\texttt{DashboardController} & Navegação principal e gestão de temas \\
\texttt{HomeController} & Dashboard com gráficos e resumos \\
\texttt{MealsController} & Registo de refeições e atalhos \\
\texttt{HydrationController} & Gestão de água com alertas \\
\texttt{ExerciseController} & Registo de exercícios com feedback \\
\texttt{HistoryController} & Tabelas filtráveis e exportação PDF \\
\texttt{SettingsController} & Configurações de tema \\
\texttt{LoginController} & Seleção de perfil \\
\texttt{RegisterController} & Criação e edição de perfis \\
\texttt{FoodDatabaseController} & Gestão de base de dados de alimentos \\
\bottomrule
\end{tabular}
\caption{Controladores da Aplicação}
\end{table}

\subsection{Persistência de Dados}
Para garantir que os dados do utilizador não se perdem ao fechar a aplicação, implementámos um sistema de persistência baseado em \textbf{Serialização de Objetos Java}.
A classe \texttt{DataStore} é responsável por gravar o objeto \texttt{AppState} (que contém toda a árvore de objetos da aplicação) num ficheiro binário local (\texttt{appstate.dat}).

\newpage

\section{Implementação Detalhada}
Nesta secção, analisamos em detalhe os principais componentes do sistema.

\subsection{Pacote \texttt{app.model}}

\subsubsection{Classe \texttt{UserProfile}}
Esta é a classe central do domínio. Representa um utilizador e agrega toda a sua informação.
\begin{itemize}
    \item \textbf{Atributos}: Nome, idade, peso, altura, género, nível de atividade.
    \item \textbf{Enums}: \texttt{Gender} (MALE, FEMALE) e \texttt{PhysicalActivityLevel} (SEDENTARY a EXTRA\_ACTIVE).
    \item \textbf{Listas}: Mantém listas de \texttt{MealEntry}, \texttt{WaterEntry}, \texttt{ExerciseEntry}, \texttt{WeightEntry} e \texttt{Food}.
    \item \textbf{Métodos Principais}:
        \begin{itemize}
            \item \texttt{getDailyCalorieGoal()}: Calcula calorias usando Mifflin-St Jeor.
            \item \texttt{getBMI()}: Calcula o Índice de Massa Corporal.
            \item \texttt{getDailyWaterGoalMl()}: Retorna 35ml × peso.
            \item \texttt{getDailyProteinGoalGrams()}: 20\% das calorias ÷ 4.
            \item \texttt{getDailyCarbsGoalGrams()}: 50\% das calorias ÷ 4.
            \item \texttt{getDailyFatGoalGrams()}: 30\% das calorias ÷ 9.
        \end{itemize}
\end{itemize}

\begin{lstlisting}[language=Java, caption=Cálculo de Calorias Diárias em UserProfile.java]
public int getDailyCalorieGoal() {
    // Equacao de Mifflin-St Jeor
    double bmr = (10 * pesoKg) + (6.25 * alturaCm) - (5 * idade);
    if (gender == Gender.MALE)
        bmr += 5;
    else
        bmr -= 161;

    double multiplier = (physicalActivityLevel != null) 
        ? physicalActivityLevel.getMultiplier() : 1.2;
    return (int) (bmr * multiplier);
}
\end{lstlisting}

\subsubsection{Classe \texttt{ExerciseEntry}}
Representa um registo de exercício físico.
\begin{itemize}
    \item \textbf{Atributos}: timestamp (\texttt{LocalDateTime}), type (String), durationMinutes (int), caloriesBurned (int).
    \item \textbf{Serializable}: Permite persistência junto com o \texttt{UserProfile}.
\end{itemize}

\subsection{Pacote \texttt{app.ui.controller}}

\subsubsection{Classe \texttt{DashboardController}}
Este controlador gere a navegação principal e a personalização visual.
\begin{itemize}
    \item \textbf{Navegação}: Carrega vistas dinamicamente num \texttt{StackPane}.
    \item \textbf{Modo Arco-íris}: Utiliza \texttt{Timeline} para animar gradiente HSB.
    \item \textbf{Modo Estático}: Aplica cor fixa com fonte personalizável.
\end{itemize}

\begin{lstlisting}[language=Java, caption=Animação Arco-íris em DashboardController.java]
private void startRainbowAnimation() {
    rainbowTimeline = new Timeline(
        new KeyFrame(Duration.millis(100), e -> {
            double hue = (System.currentTimeMillis() % 10000) / 10000.0 * 360;
            String color1 = String.format("hsb(%.0f, 20%%, 95%%)", hue);
            String color2 = String.format("hsb(%.0f, 20%%, 90%%)", (hue + 40) % 360);
            contentArea.getScene().getRoot().setStyle(
                String.format("-fx-background-color: linear-gradient(...);",
                    color1, color2, currentFont));
        }));
    rainbowTimeline.setCycleCount(Animation.INDEFINITE);
    rainbowTimeline.play();
}
\end{lstlisting}

\subsubsection{Classe \texttt{HydrationController}}
Responsável pela monitorização de água com alertas de gamificação.

\begin{lstlisting}[language=Java, caption=Alertas de Gamificação em HydrationController.java]
private void addWater(double ml) {
    user.getWaters().add(new WaterEntry(ml));
    store.save(state);
    updateView();

    if (user.getWaterConsumedToday() >= user.getDailyWaterGoalMl()) {
        Alert alert = new Alert(AlertType.INFORMATION);
        alert.setTitle("Parabens!");
        alert.setHeaderText("Objetivo de Hidratacao Atingido!");
        alert.setContentText("Excelente trabalho! Mantem-te hidratado!");
        alert.showAndWait();
    }
}
\end{lstlisting}

\subsubsection{Classe \texttt{HomeController}}
Gere o dashboard principal com múltiplos gráficos:
\begin{itemize}
    \item \texttt{PieChart}: Distribuição de macronutrientes consumidos.
    \item \texttt{LineChart}: Evolução do peso.
    \item \texttt{BarChart}: Calorias queimadas por dia (últimos 7 dias).
    \item \texttt{ProgressBar}: Indicadores visuais para cada meta.
\end{itemize}

\subsubsection{Classe \texttt{HistoryController}}
Implementa a funcionalidade de relatórios com:
\begin{itemize}
    \item \texttt{TableView} com dados filtráveis.
    \item \texttt{FilteredList} para pesquisa dinâmica.
    \item Exportação PDF usando biblioteca OpenPDF.
\end{itemize}

\newpage

\section{Funcionalidades Implementadas}

\subsection{Cálculo de Metas Nutricionais}
\begin{table}[H]
\centering
\begin{tabular}{@{}lp{8cm}@{}}
\toprule
\textbf{Meta} & \textbf{Fórmula} \\
\midrule
TMB (Homem) & $(10 \times peso) + (6.25 \times altura) - (5 \times idade) + 5$ \\
TMB (Mulher) & $(10 \times peso) + (6.25 \times altura) - (5 \times idade) - 161$ \\
Calorias Diárias & $TMB \times multiplicador\_atividade$ \\
Água Diária & $35ml \times peso$ \\
Proteína & $(calorias \times 0.20) \div 4$ kcal/g \\
Hidratos & $(calorias \times 0.50) \div 4$ kcal/g \\
Gordura & $(calorias \times 0.30) \div 9$ kcal/g \\
\bottomrule
\end{tabular}
\caption{Fórmulas de Cálculo de Metas}
\end{table}

\subsection{Multiplicadores de Atividade Física}
\begin{table}[H]
\centering
\begin{tabular}{@{}lc@{}}
\toprule
\textbf{Nível} & \textbf{Multiplicador} \\
\midrule
Sedentário & 1.2 \\
Levemente Ativo & 1.375 \\
Moderadamente Ativo & 1.55 \\
Muito Ativo & 1.725 \\
Extremamente Ativo & 1.9 \\
\bottomrule
\end{tabular}
\caption{Níveis de Atividade Física}
\end{table}

\subsection{Atalhos Rápidos de Alimentos}
\begin{table}[H]
\centering
\begin{tabular}{@{}lcccc@{}}
\toprule
\textbf{Alimento} & \textbf{kcal/100g} & \textbf{Proteína} & \textbf{Hidratos} & \textbf{Gordura} \\
\midrule
Arroz & 130 & 2.7g & 28g & 0.3g \\
Massa & 131 & 5g & 25g & 1.1g \\
Batata & 87 & 1.9g & 20g & 0.1g \\
Leite & 47 & 3.4g & 4.9g & 1.5g \\
Ovos & 155 & 13g & 1.1g & 11g \\
Pão & 265 & 9g & 49g & 3.2g \\
\bottomrule
\end{tabular}
\caption{Valores Nutricionais dos Atalhos}
\end{table}

\newpage

\section{Estratégia de Testes}
A qualidade do software foi assegurada através de uma estratégia de testes mista.

\subsection{Testes Unitários}
\begin{itemize}
    \item \textbf{Cálculo de Calorias}: Verificação da fórmula de Mifflin-St Jeor.
    \item \textbf{Cálculo de IMC}: Validação para diferentes combinações peso/altura.
    \item \textbf{Gestão de Listas}: Adição e remoção de registos.
\end{itemize}

\subsection{Testes de Integração}
\begin{itemize}
    \item \textbf{Persistência}: Gravação e carregamento de \texttt{AppState}.
    \item \textbf{Navegação}: Transição entre todos os ecrãs.
    \item \textbf{Temas}: Alternância entre modo arco-íris e estático.
\end{itemize}

\subsection{Testes de Usabilidade}
Realizados pela equipa e utilizadores externos para validar:
\begin{itemize}
    \item Clareza das mensagens e labels.
    \item Intuitividade da navegação.
    \item Feedback visual adequado.
\end{itemize}

\newpage

\section{Propostas de Trabalho Futuro}
\begin{enumerate}
    \item \textbf{Sincronização na Nuvem}: API REST com base de dados remota.
    \item \textbf{Aplicação Móvel}: Desenvolvimento para Android/iOS.
    \item \textbf{Gamificação Avançada}: Medalhas, níveis e desafios semanais.
    \item \textbf{Leitura de Códigos de Barras}: Integração com OpenFoodFacts.
    \item \textbf{Módulo de Receitas}: Criação e partilha de receitas.
    \item \textbf{Inteligência Artificial}: Sugestões personalizadas de refeições.
\end{enumerate}

\newpage

\section{Conclusão}
O projeto ``A Minha Dieta'' revelou-se um desafio enriquecedor e uma excelente oportunidade para consolidar os conhecimentos de Programação Orientada a Objetos.

Conseguimos desenvolver uma aplicação funcional, robusta e com uma interface moderna, cumprindo todos os objetivos propostos. A adoção do padrão MVC foi crucial para manter o código organizado. As funcionalidades de gamificação (alertas motivacionais), personalização visual (modo arco-íris) e o módulo de exercício físico acrescentaram valor significativo ao produto final.

A aplicação demonstra uma aplicação prática de conceitos como:
\begin{itemize}
    \item \textbf{Encapsulamento}: Dados protegidos com getters/setters.
    \item \textbf{Enums}: \texttt{Gender} e \texttt{PhysicalActivityLevel} com comportamento.
    \item \textbf{Serialização}: Persistência automática de objetos.
    \item \textbf{Padrões de Design}: MVC para separação de responsabilidades.
    \item \textbf{JavaFX}: Interface gráfica moderna com animações.
\end{itemize}

\newpage

\section{Bibliografia e Referências}

\begin{itemize}
    \item \textbf{Documentação JavaFX}: \url{https://openjfx.io/}
    \item \textbf{OpenPDF Library}: \url{https://github.com/LibrePDF/OpenPDF}
    \item \textbf{Mifflin-St Jeor Equation}: Mifflin, M. D., et al. ``A new predictive equation for resting energy expenditure in healthy individuals.'' \textit{The American Journal of Clinical Nutrition}, 1990.
    \item \textbf{Ícones}: \url{https://www.flaticon.com/}
    \item \textbf{StackOverflow}: \url{https://stackoverflow.com/}
    \item \textbf{Material de Apoio da UC}: Slides e exemplos fornecidos pelos docentes.
\end{itemize}

\end{document}
